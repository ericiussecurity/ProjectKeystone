\documentclass[../main.tex]{subfiles}
\begin{document}
\section{Appendix C: Asset Management Policy}
\subsection{Overview and Purpose}
\begin{enumerate}
    \item The purpose of this policy is to ensure that IT and information security support \CompanyName{} and allow it to win in the marketplace by establishing guidelines for managing assets both
    informational, virtual, and physical.
    \item This policy establishes the principles by which \CompanyName{} manages its assets including but not limited to computers, servers, software, information, and other critical assets.
\end{enumerate}
\subsection{Scope}
This policy applies to entities as outlined in the Cybersecurity and IT Policy base document. Specifically, this document outlines applies to all information assets owned, leased,
or otherwise used by \CompanyName{} or which store \CompanyName's information. Examples of IT assets include hardware such as user devices including endpoints and/or phones, office assets
(cameras, printers, access points), software, various as-a-service assets such as Infrastructure (IaaS) and Software (SaaS) platforms like Google Workspaces, AWS, GitHub, QuickBooks, Salesforce, etc.
\subsection{Enforcement}
See enforcement clause in the Cybersecurity and IT Policy.
\subsection{Roles and Responsibilities}
\begin{enumerate}
    \item The CISO or equivalent person designated in writing is responsible for proper implementation of the Asset Management Policy and accountable for driving the program.
    \item The IT Team is responsible for maintaining accurate inventories and tracking systems for all assets and for deploying, operating, and maintaining asset management tools.
    \item Directors are responsible for knowing their own inventories and keeping the IT Team and CISO informed of adds and changes
\end{enumerate}
\subsection{Acronyms and Terms}
Asset management - The process by which our organization will identify, inventory, maintain, and dispose of physical and virtual assets.
\subsection{Policy}
\begin{enumerate}
    \item Inventory of Computational Assets.
    \begin{enumerate}
        \item The IT Team, with the advice and assistance of the information security team, will maintain an accurate inventory of all \CompanyName's assets at all times.
        To the extent possible, \CompanyName{} will record the following information for all computation assets and may adapt this list for physical versus logical assets:
        \begin{enumerate}
            \item 1.	Device name
            \item Device serial number (or equivalent unique identifier for virtual servers)
            \item Device IP addresses
            \item Device MAC addresses
            \item Device FQDN
            \item Firmware and/or Operating system Version and date
            \item Date of purchase and/or deployment
            \item Date when the asset will exceed end of life, warranty, or support
            \item Date of when service contracts will expire
            \item Date of when \CompanyName{} must give notice by to avoid auto renew if applicable
            \item Date installed
            \item Classification level
            \item Department or unit
            \item Physical location
            \item Point of contact
            \item Brief description of role
            \item In addition to an inventory, \CompanyName{} will maintain a map, diagram, or other pictural representation of how assets are dispositioned or segregated physically and logically.
        \end{enumerate}
    \end{enumerate}
    \item The Asset Inventory can be found here: \AssetInventory
    \item The Network Map can be found here: \NetworkDiagram
\end{enumerate}

\subsection{Inventory of Software and Services}
\begin{enumerate}
    \item The IT Team, with the advice and assistance of the information security team, will maintain an accurate inventory of all of \CompanyName's software and services.
    To the extent possible \CompanyName{} will record the following information for all software, services, or IT related subscriptions:
    \begin{enumerate}
        \item Vendor name
        \item Internal point of contact
        \item Vendor point of contact
        \item Type of software, service, or subscription
        \item Version number
        \item Account Number (such as for the license or with the vendor)
        \item Quantity
        \item Cost
        \item Billing Period
        \item Expiration or renewal date
        \item End of life or support
        \item Date of when service contracts will expire
        \item Date of when \CompanyName{} must give notice to avoid auto-renewal if applicable
        \item Department or unit
        \item Description
    \end{enumerate}
    \item The software and services inventory should include IaaS and SaaS services, software deployed on servers or user workstations, both open-source and proprietary, services hosted
    by third parties, domain names, and other related registrations or subscriptions.
    \item For development processes, the development department leader is responsible for tracking all software dependencies and packages in use throughout the development and production
    pipeline and environments and reporting those inventories to IT for tracking. This supports the concept of a software bill of materials (SBOM), allowing the team to understand inventory risk.
    \item The Software and Services inventory can be found here: \AssetInventory
\end{enumerate}

\subsection{Inventory of Critical Accounts}
\begin{enumerate}
    \item Critical or privileged accounts, either for users, administrators, or automation, will be tracked and managed with the same scrutiny as other assets and will be considered
    critical to \CompanyName{} accomplishing its mission and winning in the market.
    \item Critical accounts include but are not limited to any key provided to a third party to access \CompanyName's systems, root or system level credentials of any kind,
    AWS root credentials, and the personal credentials of any \CompanyName{} level executive.
    \item An inventory of Critical Accounts will record:
    \begin{enumerate}
        \item Name
        \item Account name
        \item Date provisioned
        \item All people with access to that account or credential
        \item Privilege level
        \item Date of last review or audit
    \end{enumerate}
    \item The Critical Account inventory can be found here: \AssetInventory
\end{enumerate}

\subsection{Asset Provisioning (and Gold Image)}
\begin{enumerate}
    \item The IT team will establish “gold images” for all configurable assets \CompanyName{} relies on.
    \item \CompanyName{} will establish a gold image for at least each of its servers, baseline container images, and user workstations and will review each gold image when it is changed
    or updated to ensure that any vulnerabilities or security issues are properly documented and mitigated. Each image will be reviewed at least annually.
    \item When a new asset is acquired or deployed, it will be configured using \CompanyName's gold images and confirmed before it is deployed to development, staging, or production environments.
    IT will manage code, scripts, or other automated means to deploy gold images, set initial configurations, and provide updates to the extent possible.
    \item \CompanyName's gold images are stored and documented here: LINK\_TO\_GOLD\_IMAGE\_LIST
\end{enumerate}
\subsection{Configuration Management}
\begin{enumerate}
    \item Configuration management drives patch management
    \item \CompanyName{} has a vested interest in investing in the care and feeding (lifecycle management) of its assets. Using Gold Images is the correct place to start.
    Keeping those Gold Images and all their deployed subordinates up to date is nontrivial.
    \item Patching is hard.
    \item An IT Team with a solid configuration management program manages the Gold Images and deploys configuration updates through that management. This means that patches happen as an outcome
    of a define-one-deploy-many configuration management program rather than a box-to-box patching program.
\end{enumerate}
\subsection{Tools for Asset Control}
\begin{enumerate}
    \item The IT team, with the advice and assistance of the information security team and relevant third parties, will establish tools for monitoring the state and health of assets.
    Such tools will be deployed to all \CompanyName{} assets which can receive them and will be used to track:
    \item What asset is active and where (at least to a logical, network level)
    \item What the asset's running state is
    \item What accounts, services, or personas are leveraging the asset
    \item What software, services, or libraries are present or active on the asset
    \item Where possible, asset control tools will also provide end point protection and enable the collection of logs and other artifacts relevant to asset management and incident response.
    \item Further documentation regarding \CompanyName's asset management tools can be found here: LINK\_TO\_EDR\_OR\_RMM\_TOOLS\_DOCS
\end{enumerate}
\end{document}