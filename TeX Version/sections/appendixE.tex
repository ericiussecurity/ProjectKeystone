\documentclass[../main.tex]{subfiles}
\begin{document}
\section{Appendix E: Change Management and Change Control Policy}
\subsection{Policy Statement}
This policy exists to ensure that we adequately plan for, document, react to, and learn from changes we make to our operating posture.
\subsection{Purpose}
The purpose of this policy is to ensure that information security increases our ability to win in the market and against our competition.
\subsection{Definition}
Change Management refers to a formal process for making changes to IT systems. The goal of change management is to increase awareness and understanding of proposed changes across an
organization and ensure that all changes are made in a thoughtful way that minimizes negative impact to services and customers.
\subsection{Scope}
\begin{enumerate}
    \item This policy applies to entities as outlined in the Cybersecurity and IT Policy base document.
    \item All Changes to IT services must follow a structured process to ensure appropriate planning and execution.
\end{enumerate}
\subsection{Steps}
\begin{enumerate}
    \item Planning: Plan the change, including the implementation design, schedule, communication plan, test plan, and roll back plan.
    \item Evaluation: Evaluate the change, identify the nature of the change and what assets could be impacted, quantify the impact of an outage to those assets, determine risk level,
    and select the appropriate change type and change process to use.
    \item Review: Review change plan with peers and/or Change Advisory Board as appropriate to the change type.
    \item Approval: Obtain approval of change by management or other appropriate change authority as determined by change type. Communication: Communicate about changes with the
    appropriate parties (targeted or organization-wide). For example, any change that could impact users must be communicated well in advance!
    \item Testing: Test proposed changes in a realistic environment and prepare for fallback or rollback procedures. Capture lessons learned and build runbooks where appropriate.
    The perfect example of a change that requires a runbook is for user-impacting changes. Build runbooks for the help desk to use to solve friction during rollout.
    \item Implementation: Implement the change while anticipating a need to fallback or revert in the event of an outage. For large system changes and for user-impacting changes,
    implement a phased rollout. For example, when rolling out multifactor authentication, test rollout with the security team first, and then a subset of privileged users,
    and then a subset of unprivileged users, and then by department.
    \item Documentation: Document the change and any review and approval information. The documentation phase will include capturing feedback from the following:
    \begin{enumerate}
        \item Departments or persons negatively impacted by the change
        \item The teams responsible for rolling out the change
        \item Stakeholders with significant investment into the change
        \item Any emergency personnel or teams pulled in to aid the rollout
        \item Post-change review: Build the foundation that enables better, faster change in the future.
        \item This phase may be the most important phase of the change process.
        \item This phase is designed to streamline further organizational change.
        \item This phase allows future change agents a platform to better understand the organization and infrastructure which enables them to build a better plan and avoid previous, known pitfalls.
    \end{enumerate}
\end{enumerate}
\subsection{Authority}
Department heads are responsible for the change management process within their department and responsible for coordinating and deconflicting with change outside their department. For measures that require additional funding, the department heads must consult Finance during the Approval step. 
\subsection{Documentation}
\begin{enumerate}
    \item a.	All Normal and Emergency changes, evaluations and approvals will be documented to allow customers to understand what was changed, the reason it was done, and the process that was used to make a change. The following details the kind of information that will be logged for each change and where it will be logged.
    b.	Change Log
    \begin{enumerate}
        \item All Standard, Normal, and Emergency changes are logged in the Change Log
        i\item The Change Log contains: 
        \begin{enumerate}
            \item Who made the change 
            \item What was changed 
            \item Why the change was made (Reason/Comment)
            \item And when the change was made
        \end{enumerate}
        \item Process Log
        \begin{enumerate}
            \item Normal Medium, Normal High, and Emergency changes are logged in the Process Log
            \item The Process Log contains
            \begin{enumerate}
                \item Test Plan and testing results 
                \item Risk assessment documentation 
                \item Communication Plan 
                \item Deployment Plan, including back-out contingencies
            \end{enumerate}
        \end{enumerate}
    \end{enumerate}
\end{enumerate}

\end{document}