\documentclass[../main.tex]{subfiles}
\usepackage{enumitem}
\begin{document}
\section{Appendix A: Information Classification, Handling, and Retention Policy}
\subsection{OVERVIEW AND PURPOSE}
\begin{enumerate}
    \item \CompanyName{} shall adopt and follow well-defined and time-tested plans and procedures to ensure that sensitive or critical information is classified correctly and
    handled according to \CompanyName's policies. The purpose of this policy is to help people understand what information may be used where and shared with whom and utilize that
    restriction to protect from unauthorized use and disclosure. This policy helps to facilitate how the \CompanyName{} team identifies information to support routine disclosure and
    active dissemination of information, which also helps protect \CompanyName's intellectual property.
    \item This allows \CompanyName{} to maintain continuous business objectives properly steward information on behalf of relevant parties.
    \item Information is considered a primary asset for \CompanyName. \CompanyName{} uses multiple types of information assets, and the sensitivity and handling requirements of these
    information assets may vary.
\end{enumerate}
\subsection{Scope}
This policy applies to entities outlined in the Cybersecurity and IT Policy base document. Specifically, this document outlines information and data assets.
\subsection{Enforcement}
See enforcement clause in the Cybersecurity and IT Policy.
\subsection{Roles and Responsibilities}
The CISO, DPLO, Head of Internal Audit, or equivalent person designated in writing is responsible for maintaining this policy in conjunction with appropriate other stakeholders such as
the Legal Counsel. This role is accountable for proper implementation of the Information Classification and Handling Policy with various other department leads assuming responsibility for
implementing the policy within their respective departments.
\subsection{Acronyms}
None for this policy
\subsection{Policy}
Privacy: This policy inherits its privacy from the base policy. Portions of this policy may, by exception, be held at a higher, more restrictive classification level and stored separately
in accordance with access control policies as governed by that more restrictive classification level.
\subsection{Classification Categories}
\begin{enumerate}
    \item \CompanyName{} categorizes information into three classes: Public, Internal, and Confidential.
    \begin{enumerate}
        \item Public - Definition: Information or assets intended for disclosure to or interaction with the public or that, if disclosed, pose no risk or damage to \CompanyName.
        \begin{enumerate}
            \item Public information includes information assets that do not have any confidentiality or regulatory requirements, or that can be disseminated to the public.
            \item Examples include press releases, annual financial reports in accordance with compliance, marketing material, social media, and \CompanyName's website.
        \end{enumerate}
        \item Internal - Definition: Information or assets that are not intended for public release but do not necessarily pose a significant risk or damage to \CompanyName{} if released.
        \begin{enumerate}
            \item Internal information includes information necessary for the organization and operation of \CompanyName{} that is not necessarily confidential or information that can be circulated
            freely within all offices or departments in \CompanyName{} but not necessarily the public.
            \item Examples include personnel assignments, office orders, internal circulars, movement of personnel or equipment, or invoices.
        \end{enumerate}
        \item Confidential - Definition: Information or assets that pertain to the specific needs of the project, team, department, or business process or that pose a significant risk or
        damage to \CompanyName{} if improperly disclosed.
        \begin{enumerate}
            \item Confidential information includes information necessary for the business operations of departments or units, information that cannot be freely circulated within \CompanyName,
            PII or PCI data, sensitive or proprietary information, and intellectual property that cannot be publicly disclosed except then directed by law, regulation, or legal order.
            Confidential information shall be restricted to need-to-know (an element of least privilege) such as those entities deeply associated with a project or business process.
            Access must be backed up and archived. It must be encrypted when transmitted or password-protected when encryption is not possible. It must also be backed up and archived when stored.
            \item Examples include any information that would reduce \CompanyName's ability to go to market and win, such as business strategy, plans for mergers or acquisition, PCI data,
            personnel files, service or other agreements, or other techniques or procedures not appropriate for public release.
        \end{enumerate}
        \item Confidential: Special, Related to XYZ: This is a sub-category of confidential that requires special handling potentially for client, compliance, or legal reasons outside
        of standard Confidential categories. Further definitions should be stored at a separate classification level than in this document.
    \end{enumerate}
\end{enumerate}
\subsection{Secure Handling of Information Assets}
\begin{enumerate}
    \item All information will be labeled according to its classification label in the header and footer of the document and clearly within the file name. In the case of physical storage media,
    it will be physically labeled with the highest level of information stored therein.
    \item Some classification levels may require approval from the CISO, CEO, or legal counsel prior to transmission.
    \item Restrict mailing and/or shipment of confidential information through only trusted mail services or couriers who must show authentication.
    \item Store hard copy confidential information behind double lock and key.
    \item Hard copy confidential information must be shredded at the end of life.
    \item Take prudent cautions to prevent unauthorized personnel from accessing higher levels of restricted information.
    \item Limit access control both by role and by person/entity.
    \item Similarly treat confidential information as you would with both chains of custody and custodial responsibility.
    \item Encrypt spooled data and validate user identity and permission prior to printing.
    \item Limit distribution to “need to know” and “need to use.”
    \item Review persona and entity access monthly and at the end of major project phases.
    \item The approval authority for downgrading an information asset's classification level resides with department leads, and when a classification level is in question,
    department leads are to consult with the owner of this policy, legal counsel, and or other executive stakeholders.
    \item See associated charts, process documents, and controls more granularly here: \InfoInventory
\end{enumerate}
\subsection{Classification by Aggregation}
\begin{enumerate}
    \item Aggregating data and or information can change the classification level of that data. Converting data into information can change the classification of
    this newly created information asset. This is because information aggregation both represents an intellectual property base and can provide insights into the \CompanyName’s internal
    workings that could cause the \CompanyName{} to lose in a competitive market.
    \item When information is aggregated, it will, at a minimum, be classified at the highest level amongst all the information aggregated. For example, when information of internal
    classification is aggregated with information of confidential classification, the aggregated information is controlled as CONFIDENTIAL.
    \item If several items at the same classification level are aggregated, the aggregated information will instead be classified as one level higher than the information aggregated.
    For example, several pieces of internal information aggregated into one document will be controlled as confidential.
\end{enumerate}
\end{document}