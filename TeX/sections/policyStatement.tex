\documentclass[../main.tex]{subfiles}
\usepackage[shortlabels]{enumitem}
\begin{document}
\section{Cybersecurity and IT Policy}
  \subsection {Policy Statement}
   \CompanyName shall adopt and follow well-defined and time-tested plans and procedures and layout cybersecurity objectives to meet business objectives and ensure continuity of
   operations and processes.
  \subsection{Purpose}
  \begin{enumerate}
    \item The purpose of this policy is to establish cybersecurity guidelines that ensure our ability to accomplish our mission: \MissionStatement.
    \item Put another way; this program allows us to go to market and win by enabling us to create and capture more revenue, reducing cybersecurity risk, which reduces expected
    loss from damage to reputation and cost to recover.
    \item This cybersecurity policy explains why we need to proactively steward information, technology, and people. We build and maintain our security program because it enables us to
    build trust, confidence, and goodwill within the market and with our teammates and partners. Further, we specifically outline the program standards to which we aspire and outline
    the processes by which we achieve those standards such that we can both capture more revenue in the market and reduce overall risk to the organization.
  \end{enumerate}
  \subsection{Scope}
  \begin{enumerate}
      \item This policy applies to \CompanyName, our employees, our applicable third parties, our technology use, our IT infrastructure, our computing resources, our networks, our information,
      and our data, and any of the above categories that we could be perceived as being stewards over.
      \item The scope is limited by applicable legal and contractual obligations where required.
  \end{enumerate}
  \subsection{Enforcement and Expectation of Compliance}
    Any user or person bound to this policy that is found to have violated these policies may be subject to disciplinary action, including termination of employment, contract, or another arrangement.
  \subsection{Roles and Responsibilities}
  \begin{enumerate}
    \item You - As a \CompanyName staff member or applicable third-party, you are an integral part of our security program. You maintain appropriate levels of access and privileges and defend
    the \CompanyName with constant vigilance and discipline. You are required to keep the appropriate authorities informed of technology adds, changes, and reductions and report vulnerabilities
    and gaps. We cannot protect your needs if we are unaware that you have something you need us to protect.
    \item CEO - The CEO is responsible for living cybersecurity fundamentals and setting the standard which empowers the team to drive the program. The CEO establishes priorities and goals
    for \CompanyName, which the cybersecurity program and policies support.
    \item CISO - The CISO leads the cybersecurity department and is responsible for reducing business risk through the judicious application of training, auditing, compliance, governance,
    and communication across \CompanyName and is responsible for establishing all cybersecurity policies subordinate to this one, including but not limited to the policies attached to this
    policy as appendices. When applicable, the CISO receives policy and program guidance from the Head of Internal Audit. The CISO role may, from time to time, include third parties such as
    consultants or virtual CISOs.
    \item Legal Counsel - \CompanyName legal counsel provides a checkpoint for all security programs and ensures that areas such as data retention or emergency mode operations comply with legal
    and regulatory requirements. They are a primary stakeholder during some forms of incident response. They are the direct voice of the \CompanyName between internal stakeholders, any government
    entity, and partner, client, or user communications. Often, the Legal Counsel works closely with the external (public) communications team during an incident.
    \item Head of Internal Audit - The head of internal audit is responsible for establishing \CompanyName's top-level policies, including this policy, and for reviewing and ensuring the proper
    implementation of \CompanyName's top-level policies and all policies nested underneath them.
    \item IT Team - Execute change in harmony with other stakeholders identified here to maintain the technical ability to execute \CompanyName operations. Support the needs of \CompanyName's
    other staff, lead the change control process, and govern identity and access management through the organization. The IT team includes badged employees and contracted or third-party service
    providers who directly manage \CompanyName's assets and infrastructure.
    \item Office Manager - Responsible for physical access management, key control, and daily security. The office manager coordinates responses to physical incidents and notifies cybersecurity
    and IT personnel.
    \item Privileged Users - Users of \CompanyName IT and computing assets with administrative or increased access and permissions. \CompanyName places a high degree of trust and confidence in
    its privileged users because they have access to more information and resources than other users and can grant others access to information and resources.
    \item Third-Party Stakeholders - People or entities outside the organization that require access to \CompanyName's information. Those stakeholders, as applicable, must sign proper authorizations
    such as non-disclosure agreements (NDAs/MNDAs), service agreements (SLAs, GLAs, etc.), and Business Associate Agreements (BAAs) when applicable before receiving privileged access. They are
    required to maintain the same levels of diligence to protect the confidentiality, integrity, and availability of our information, systems, and resources.
    \item See associated RACI Charts, process documents, and personnel documents that define roles and responsibilities more granularly here: \RACIDocs
    \begin{itemize}
      \item Note: RACI stands for Responsible, Accountable, Consult, and Inform. RACI documents are used to outline who owns what portions of processes and how to process information is communicated.
    \end{itemize}
  \end{enumerate}
  \subsection{Terms}
      Asset - Including but limited to any intellectual property, information, data, physical or virtual property, equipment, software, service, or platform owned, operated, maintained or
      leased by \CompanyName. In some contexts, this could include personal devices that have access to \CompanyName  assets.
  \subsection{Policies}
  \begin{enumerate}
    \item \CompanyName's Head of Internal Audit establishes this Cybersecurity and IT policy and delegates responsibility for all policies nested under it, such as the appendices attached herein.
    \item \CompanyName strives not just to comply with regulatory frameworks—such as PCI and GDPR where applicable—but to improve and sustain its operations and ability to serve clients and users
    through deliberate risk-management. To this extent, we have an ongoing cybersecurity and risk-management program and will continue to apply and mature it both in scope and in practice.
    \item We will employ countermeasures that include but are not limited to:
    \begin{enumerate}
      \item Setting program standards
      \item Risk identification and classification
      \item Security training
      \item Identity management and identity-focused security
      \item Change control
      \item Judicious use of the principle of least privilege (applied zero trust)
      \item A focus on SaaS-based applications
      \item Deliberate management of third-party risk
      \item Securing communications, data, information, infrastructure, systems, endpoints, and owned, leased, or open-sourced platforms.
      \item Out-of-band communications
      \item Information classification, retention, and disposal
      \item External audit, assessment, and testing when and where applicable (usually annually)
      \item Asset, Vulnerability, and Patch Management
      \item Incident Response and Incident Management
      \item Log and Event Management and Retention
      \item Our strategy is simple. As Dustin Wilcox says: “Minimize the Attack Surface, Complicate Unauthorized Access, Rapidly Detect, Respond to, and Contain Incidents.”
    \end{enumerate}
  \end{enumerate}
  \subsection{Exceptions}
  \begin{enumerate}
    \item Deviations from or exceptions from these policies shall be submitted to the CISO and IT Team for approval and documentation.
    \item The CISO and IT Team maintain a repository of approved exemptions and associate risk here: \ExceptionsRegistry
    \item Note: Access to the exemption repository is controlled based on role and need to know. The link will not work for you without either.
  \end{enumerate}
  \subsection{Documentation and Control}
  \begin{enumerate}
    \item Documents
    \begin{enumerate}
        \item This Cybersecurity and IT policy document, all policies or documents contained or referenced herein, and all \CompanyName policies shall be controlled. Version control will
        be applied to distinguish current versions from all previous revisions. All such policies and documents will be retained in digital form for two years from their last effective date
        unless other authorities mandate the period.
        \item After two years from the last effective date of a policy or document, all physical or digital copies of a document or policy will be securely destroyed by shredding and/or secure
        deletion except when a legal hold or other relevant exception is in place.
        \item This policy does not apply to log retention, which is governed by a separate policy.
      \end{enumerate}
  \item{Records}
  \begin{enumerate}
    \item Records generated as part of the Cybersecurity and IT policy or a policy contained or referenced herein shall be retained for two years. Department leads will own their respective
    policies and will audit yearly. Records shall be encrypted at rest and in transit.
    \item To maintain compliance, specific records with a longer retention rate will be maintained for the period outlined by the appropriate laws and statutes. These include but are not
    limited to legal actions, insurance settlements, and tax records that have varying retention requirements.
  \end{enumerate}
  \item{Distribution and Maintenance}
  \begin{enumerate}
    \item The Cybersecurity and IT policy is not a public document. However, it may be provided to all badged stakeholders and a limited set of third parties when applicable.
    This availability includes all changes and revisions. The Head of Internal Audit, DPLO, and or CISO will be responsible for this document and its contents.
    \item The CISO and IT Team maintain a repository of approved policies, both current and historical here: \DocumentationandPolicies
  \end{enumerate}
  \end{enumerate}
  \subsection{Subordinate Policies (Appendices)}
  \begin{enumerate}
    \renewcommand{\labelenumi}{\Alph{enumi}.}
    \item Information Classification, Handing, and Retention Policy
    \item Incident Management Policy and Process
    \item Asset Management Policy
    \item Acceptable Use Policy
    \item Change Management and Change Control Policy
    \item Remote Working Policy - See Employee Handbook
    \item Security Awareness Policy and Process
    \item Business Continuity Policy and Plan
    \item Technical Security Policy
    \item Reputation Management
    \item System Lifecycle Management
  \end{enumerate}
\end{document}