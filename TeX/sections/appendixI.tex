\documentclass[../main.tex]{subfiles}
\usepackage{enumitem}

\begin{document}
\section{Appendix I: Technical Security Policy}
\subsection{Policy Statement}
This policy exists to ensure that we can appropriately design an overall technical security policy that can serve as a focus for our technical security strategy.
\subsection{Purpose}
The purpose of this policy is to ensure that technical security policy increases our ability to win in the market and against our competition.
\subsection{Scope}
This policy applies to entities as outlined in the Cybersecurity and IT Policy base document.
\subsection{Roles}
\begin{enumerate}
    \item Roles are inherited from the base Cybersecurity and IT Policy document. Additions are by exception.
    \item Data Loss Prevention Officer - \CompanyName can either implement the DLPO role as an additional duty or align the role independently.
    The DLPO leads the subset of risk mitigation efforts to support the CISO and reduce risk to the partner ecosystem. The DLPO is charged with
    identifying various data across the organization and classifying it (Identify), building safeguards into the higher levels of data classification (Protect),
    implementing alarms to understand when that data has left the organization's control (Detect), builds and leads the response to that data lose (Respond),
    and rebuild that data from backups and engage with stakeholders outside the organization when appropriate to limit the effects of that data loss (Recover).
    \item IT Committee - Coordinate with the CISO and DLPO to provide input to the Technical Security Policies are appropriate. The IT Committee will serve at the
    Change Control Review Board and has a serious but not deterministic voice in controls applied at 49\% of the vote versus the CISO. However, the IT Committee can veto
    projected per-user spend increase of 101\% or greater.
\end{enumerate}
\subsection{Policy Overview}
\begin{enumerate}
    \item \CompanyName will continuously work to decrease our attack surface and improve our security posture by applying the NIST fundamentals of
    Identify, Protect, Detect, Respond, and Recover to the attack surface that we cannot decrease.
    \item We will review both annually and as required by partners or other events such as but not limited to new legislation or otherwise our security compliance requirements.
    In the event that we have no formal requirement as outlined by local laws, we will apply minimum baselines as required by clients, customers, partners, suppliers, local laws and regulations,
    and reasonable prudence.
    \item We know and understand that “getting hacked” is not a matter of IF, but WHEN. Therefore, we will ASSUME COMPROMISE. This means that we will take those actions that are expected of us
    throughout our stakeholder network to ensure that we can, on any given day, answer the questions as follows:
    \begin{enumerate}
        \item Have we been breached?
        \item Have we contained the breach?
        \item Do we have a duty to notify?
        \item Do we need to bring in an external incident response or legal team?
        \item What did the breachers (attempt to) accomplish?
        \item Did we lose anything?
        \item What actions do we need to take?
        \item What is the impact and scope of the breach?
        \item How long will it take us to recover?
        \item If we ASSUME COMPROMISE, then we must continuously endeavor to IDENTIFY, PROTECT, DETECT, RESPOND, and be prepared to RECOVER.
    \end{enumerate}
    \item Identify
    \begin{enumerate}
        \item We will continuously integrate our security team with our IT and logistics and operations teams to understand WHAT nodes (devices, infrastructure, software, instances, VPCs, SAAS, etc)
        are a part of our attack surface. We will use CIS Controls 1\&2 to outline how we conduct inventory of those assets prior to bringing those assets into our attack surface and continue to
        develop our attack surface mapping following because if we know one thing it is that: “We cannot trust our inventory of assets because availability and adaptability often override
        confidentiality and integrity.”
        \item This means we must continue to place sensors to continuously, passively detect changes to our attack surface and actively scan annually externally unless required additionally by
        some other factor to understand how our attack surface may have changed without the knowledge or understanding of the security team. Often this is described as “Shadow IT.”
    \end{enumerate}
    \item Protect
    \begin{enumerate}
        \item Understanding that modern systems often have some form of “Shared Responsibility Model” means that we have less control over some systems such as SAAS than we might with other
        systems such as internally developed and hosted applications run on \CompanyName-owned hardware. However, operating within that understanding is not an excuse to assume risk.
        \item While we are strategically a SAAS-focused \CompanyName, we will not assume that full responsibility aligns with the provider and we will apply reasonably prudent controls to protect our
        data and our people. We will apply such controls as multi-factor authentication, security review, vendor risk management programs, and best practices for password policies to limit risk
        exposure to data in transit, in process, and at rest.
        \item User endpoints will be hardened with preconfigured “gold images,” will be enrolled in \CompanyName-controlled management programs (programs that balance security with personal privacy
        with preference focused on reasonable privacy), protected with multifactor authentication such as biometrics or other widely accepted factors, protected with increased defenses such as
        corporate-grade signature and behavioral analytics systems that block suspicious activity, allow-lists when possible, sandboxed email, and other zero-trust fundamentals such as restricted
        access.
    \end{enumerate}
    \item Detect
    \begin{enumerate}
        \item Given that we are a SAAS-based, remote-focused \CompanyName, our detection measures focus on endpoint, network, identity, and SAAS applications.
        \item Identity - \CompanyName will implement a zero-trust model of security that focuses on Identity. That means that many of our detection efforts are focused on identity.
        Therefore, a large portion of our efforts are focused at the identity layer and we will continue to extend our identity integration into SAAS partners such as SalesForce, M365,
        and other supporting applications as required. This provides us visibility into identity areas that we might not have otherwise understood. This way we can understand where our data is,
        what our risk exposure is, what adversaries might be attempting to or actually accessing data from compromised identities, and what actions we need to take at the identity layer to limit
        exposure. Identity logs are pushed to the data aggregation solution to allow for both post-mortems, correlation, and hunt activities.
        \item Host (endpoint) - Given that we are a remote workforce, user endpoints are one of our weakest avenues.  We can never truly understand whom has access to those devices that might
        have \CompanyName data on them. That means that any user that attempts to access \CompanyName data must register the device with \CompanyName management before accessing that data. This allows us to
        detect device health prior and therefore limit access to data for unhealthy devices. This is part of our MDM/UEM strategy. For runtime environments and/or daily operations on those devices,
        we deploy detection systems appropriate to the hardware and operating system that the user needs. Those detection systems such as host-based intrusion detection systems (HIDS) are often
        integrated into the same agents running some of the outcomes listed in the Prevent section above. Alerts can be pushed from those endpoints to admins and response teams. Endpoint logs are
        pushed to the data aggregation solution to allow for both post-mortems, correlation, and hunt activities. This is often done through Intune and the O365 Security Center.
        \item Network - For those instances when we do have an office, that office will have an employee and guest network. We will maintain appliances on that network and any remote access
        solution such as VPN to understand which devices are sending what data to whom. Logs from those appliances (both sensors, and network and security devices) are then pushed to the
        data aggregation solution to allow for both post-mortems, correlation, and hunt activities.
        \item SAAS - While not all SAAS platforms have SAAS logs, we will ingest those logs that are appropriate into our data aggregation solution to allow for both post-mortems,
        correlation, and hunt activities.
        \item Log Aggregation Policy
        \begin{enumerate}
            \item Until we have a compliance framework that requires longer storage, we will store the logs that our aggregation appliance has ingested for 13 months.
            This number of 13 months is usually more than most regulations require but we've seen multiple times adversaries that are patient enough to wait 366 days to execute their actions
            because they know most companies shred logs at 365 days.
            \item Whenever possible we will “pre-compute” logs on the node before transmission to limit data volume and velocity into the aggregation solution.
            \item Aggregation solutions need to not only have the capacity to support the incident questions outlined above such as, “Were we breached,” but also allow for correlation and hunt activity to help understand where data resides, where it moves, and what is happening to it.
            \item The timeframe here is used as a guiding principle and there may be exceptions under this non-exhaustive list: certain jurisdictions, compliance frameworks, and client contracts.
            \item Respond - See Incident Management Policy and Process
            \item Recover - See Business Continuity Policy and Plan
        \end{enumerate}
    \end{enumerate}
\end{enumerate}
\end{document}